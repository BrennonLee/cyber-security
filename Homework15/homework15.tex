\documentclass[12pt]{article}
\setlength{\oddsidemargin}{0in}
\setlength{\evensidemargin}{0in}
\setlength{\textwidth}{6.5in}
\setlength{\parindent}{0in}
\setlength{\parskip}{\baselineskip}
\usepackage{amsmath,amsfonts,amssymb, siunitx}
\usepackage{listings}
\usepackage{graphicx}
\usepackage{array}
\usepackage{fancyhdr}
\usepackage{pgfplots}
\usepackage{listings}
\usepackage{flexisym}
% \usepackage[table]{xcolor}
\usepackage[utf8]{inputenc}
\pagestyle{fancy}
%Code listing style named "mystyle"
\lstdefinestyle{mystyle}{
  basicstyle=\footnotesize,
  breakatwhitespace=false,
  breaklines=false,
  captionpos=b,
  keepspaces=false,
  numbers=left,
  numbersep=5pt,
  showspaces=false,
  showstringspaces=false,
  showtabs=false,
  tabsize=2
}
%"mystyle" code listing set
\lstset{style=mystyle}
\begin{document}
\lhead{{\bf CSCI 3403 \\ Homework 15} }
\rhead{{\bf Brennon Lee  \\ Fall 2018, CU-Boulder}}
\renewcommand{\headrulewidth}{0.4pt}
\vspace{-3mm}
\begin{enumerate}
  % QUESTION 1
  \item From the terminal (unix, mac, windows, whatever), type the command “dig +dnssec com DNSKEY”. Describe the steps you would take authenticate the DNSKEY of com (you may assume the root DNSKEY is known) \\

  \textbf{To authenticate DNSKEY of com, begin by running the command above to return the 3 com keys and RRSIG made by com private key 1. Next we verify the RSA (i.e. verify([key1,key2,key3], RRSIG, com pub key 1)). Now we need to verify com pub key 1 which we can get from the root DNSKEY (which we already know!). So we take that response and use the hash that was signed by root private key 2. `Verify("hash", RRSIG, root pub key 2)` and then hash(com pub key 1) and see if that equals the hash. Lastly, we need to verify the root pub key2 now. So running `dig +dnssec . DNSKEY` gives us 3 keys and the RRSIG signed by root private key 1. Next we call `verify([key1, key2, key3], RRSIG, root pub key 1)`. This will authenticate the DNS of com } \\

  % QUESTION 2
  \item From a terminal (unix, mac, windows, whatever), type the command “dig +dnssec baa.darpa.mil”. Describe the steps you would take authenticate the IP address of baa.darpa.mil \\

  \textbf{To authenticate the IP address of `baa.darpa.mil`, you begin by asking for the record for that domain. There is no reason to believe the response is legit so you must verify it by taking the RRSIG and the `darpa.mil`'s public key 2 to verify the signature that was signed by the parent's private key.
  But we need to verify the public key 2 so we get that via `Dig +dnssec darpa.mil DNSKEY`. The response contains three keys that we need to verify which can do something like `verify([key1, key2, key3], RRSIG, darpa.mil public key 1)`. But now we need to verify darpa.mil.public key 1 so we ask `mil` for key 1 via `dig +dnssec darpa.mil DS`. This returns a hash of the darpa.mil public key 1 which was signed with private mil key 2. So now we can take the hash and use our verify function (i.e. verify(hash, RRSIG, public mil key 2)) which tells us if the hash was correct. If public key 2 is known, you can verify the hash via `hash(darpa.mil pub key 1) === prev_has` and if that's true, the public key 1 is correct and everything has been authenticated.} \\\

  % QUESTION 3
  \item Use the dig command to obtain all the DNSKEYs you need to authenticate the darpa.mil DNSKEY. \\

  \textbf{running `dig +dnssec darpa.mil DNSKEY` gives the following 3 keys and 3 signatures: } \\

  \textbf{key1 - DNSKEY	257 3 8 AwEAAbIwRPXs66IueHkuvY4SHdiMUQQP8nZ6FM5djokqxH58sqdowtz6 Y+0vktOvAqlNkAbW0F75Qb8MoOrg1Ehn+5S/IKypOuXzY1LSi2le4I7h qjWCAIot+rlRp6oSruuEGN35qQRZdfyggaquo+XxLG2Ex1vNZagd1N15 XNFbUmu5On1ekRFl+VPAp0/bHUHFjQykwcZbwkEdxcaGq+0NhpXBYvsF itTWI3BsgGCwKG7FSyY91ICCBFKjio3XS8z6KuQ64w3Y9cCvwn3FyC1y o/uhqGDkocKwaPz+GFtpoUpGEG75IaAIwE6jrDFfwQchF8XAzHkBihO5 /mkXwATVLt8=} \\

  \textbf{key2 - DNSKEY	257 3 8 AwEAAbIwRPXs66IueHkuvY4SHdiMUQQP8nZ6FM5djokqxH58sqdowtz6 Y+0vktOvAqlNkAbW0F75Qb8MoOrg1Ehn+5S/IKypOuXzY1LSi2le4I7h qjWCAIot+rlRp6oSruuEGN35qQRZdfyggaquo+XxLG2Ex1vNZagd1N15 XNFbUmu5On1ekRFl+VPAp0/bHUHFjQykwcZbwkEdxcaGq+0NhpXBYvsF itTWI3BsgGCwKG7FSyY91ICCBFKjio3XS8z6KuQ64w3Y9cCvwn3FyC1y o/uhqGDkocKwaPz+GFtpoUpGEG75IaAIwE6jrDFfwQchF8XAzHkBihO5 /mkXwATVLt8=} \\

  \textbf{key3 - DNSKEY	257 3 8 AwEAAbIwRPXs66IueHkuvY4SHdiMUQQP8nZ6FM5djokqxH58sqdowtz6 Y+0vktOvAqlNkAbW0F75Qb8MoOrg1Ehn+5S/IKypOuXzY1LSi2le4I7h qjWCAIot+rlRp6oSruuEGN35qQRZdfyggaquo+XxLG2Ex1vNZagd1N15 XNFbUmu5On1ekRFl+VPAp0/bHUHFjQykwcZbwkEdxcaGq+0NhpXBYvsF itTWI3BsgGCwKG7FSyY91ICCBFKjio3XS8z6KuQ64w3Y9cCvwn3FyC1y o/uhqGDkocKwaPz+GFtpoUpGEG75IaAIwE6jrDFfwQchF8XAzHkBihO5 /mkXwATVLt8=} \\

  \textbf{RSSIG 1 - RRSIG	DNSKEY 8 2 302400 20181218225351 20181211220726 55599 darpa.mil. uwp02bSnGco513FjqQoA87E8RqFhD5MAwmKoPtJQsrwl7lmiPwAwCZDg WaXCMN2dfWomZb6qJZAlJ22C9c255c3JwD6OY1a9PsMUsSoIWLLnL0kO 9eygBrPpqvE95v66IIdm++yWvYYU8Z0XPZa9JIE7a5Xvk7m34+RTCm5Z VkW9G0A+M8rwbOckBOHLG0RuurGSLa0C8/EECQkOTFWB2BwVyLURsVO/ 8PVuhj66jcFBH4OZfRu6y4RpmvwbK46o9WyK9cy9f9tnoggnZXTwcj3O BpPH1ODmOtfrPzN5qFrRavwsLlIM98L1gTzTLaFAEeiqALDl9M3lWOuX bcZEPw==} \\

  \textbf{RSSIG 2 - RRSIG	DNSKEY 8 2 302400 20181218225351 20181211220726 4975 darpa.mil. mt0xo/JwYDmUqJZNIaZY+rQfdOh+lPl9zaaQerBVIqnCcNCi/shsoQUh 81/Pe0pQp6s1JONH8KIvxBC7MwsDwSrnbVlgHknzgUSeBSxf5gXTIZSU vaKfVsbWP5Ngl3UFpDNwPbhS8XmjmZBgMJdlgaWJPkFUNdVZPchgdo+E 4laSqJxxRdyH7qvwK2rhISjVfjTNPs/3CclzGWRg3SgOS3sadKVQvVyv 9/IaJsGCzpeo48ug+jEmbiMCPdNanV2JPmizKWXm/0oXsfrgmmmjZteQ RkNkcQDXGbDNc/6UqsuGJ6qgdVD5iVcE3x8QUgXk67f6AdzG/Al+cpCe qLscnw==} \\

  \textbf{RSSIG 3 - RRSIG	DNSKEY 8 2 302400 20181218225351 20181211220726 36843 darpa.mil. Wz9LeUdGEePLLZxQ6beNySdjSmkR5Z/1zsza+F6zEv8MdhqFzRJL2PhZ vZu1Ib2VW7rcZen2j6Ki0daG1bFKmDokarI1Xep0CIoFntTG3gDCD89Z xuiEJvtKjl32vMeWDnGU2lxiyliFGJVYFILbvlLtJ9jxEoibm9XY5Umb eJR2c6RCzngIpPvex+zRqR22CtbfZ1ib27hGtF1IfEQElmfA2v2dVrEs ErLZOkSRrW5c7K9JHBzIRd6LbPhpqIxdQQ3OFfHOmcCD8aVR018VNog/ 1K9VOYZ0KXfTyp+LIJdErtmWQVkNxJmwz4EHzjUh/JthZWcoHRZE8CTL uHUovg==} \\

  \textbf{Next we need mil's record which is found by: `dig +dnssec mil DNSKEY` and gives the following 5 authority responses} \\

  \textbf{RRSIG	SOA 8 0 86400 20181225210000 20181212200000 2134 . LeXTKc+2ovLnBH5cENe+5t24Ow0TE5uXqtvm87Y2YoiRGsEQKcecWnyr AoKBJ5/DGEC6q1YsiFoatbEeuuoXyTb6BvrbFtS5yyzU010+CqX4/pMD glnav9ARemc8pRtHIqOlXJpdsfGwkUO/2536xso4SSnUviN/u8CoEFEm g/HeqwkmMYwatq4k/6b+VE1oMZ5XR9SWi4WvSCNHrpuBTG5NILend2Mx Zy0l4P2KKTgpNRsaXb7iCtEpqpG+iflfd/AdU+i3K38dDYh3i1Nf398P o0niRXiC8dN1sm1avoHgY26u+O9B55zf8KjydBrzwle24uwkL0h+Eepp BDJNfQ==} \\

  \textbf{NSEC	do. NS DS RRSIG NSEC} \\

  \textbf{RRRSIG	NSEC 8 1 86400 20181225210000 20181212200000 2134 . FLfoN3VXJzpZGP6/H4lBb/h/cXyjAUMZTX5AuUgZbcGU6YO7qAXkYcrq rpvMqLO6MwE36ilzs+CDyfx2NF+HdoiqcnE197fJHTmRKhAQzajZ9EaC M8MvTiajnwMj1M5UH2HZpPwYqiDmxVSUHt96E37tJKw83O0B8nu4tb/c jjktetUSQULi09E0lHHiyEWDTujRPBXiLMcr9TARLYErMEbvNHgUsNck cl8DUmVtwkYPOad948L3dihafAH3wSb1UZg/6p83DrlaSILpr69fEsAS nzJObNiBezAAl+/lscjcwnA/gKLRdmjPMAcOnkRH9hdxxq0aGxED/Ak1 P22/Ig==} \\

  \textbf{NSEC	aaa. NS SOA RRSIG NSEC DNSKEY} \\

  \textbf{	RRSIG	NSEC 8 0 86400 20181225210000 20181212200000 2134 . pFQ5IN7eg326hrc1wRNAM5bKGITNLgjlzTPtxkYLhBsycGdp06SWhZWV 6ieiO6TKG+5UX6TmxXBFXOsI6Dj/vlX3tFYsiZxW5QPMV9X8Ijtr6fuQ khEM+BrVWdKa94Uo4uZViA5aWyMAduonR6+0RrxlDKo18x8HnNDOAA3q uWgh2HS7H8FE+vMAGn0+2fzcShR1fn45q8GUo1da876wTpm1xZY3klhW uRkSy7S0+Gwuk/hh2Py/JkSz2I82CS/VknOcgiPjmDLigUou7eXP7ISj fcjfuRv+K+FKPKWWzBTK3m7hQsJGHHSeGBV1R3zr7LPzaXHgGqG6O1yc zX1++A==} \\

  \textbf{which gives us all the keys we need to authenticate by following the steps in the previous problem.}

  % QUESTION 4
  \item From a terminal (unix, mac, windows, whatever), type the command“dig +dnssec www.darpa.mil”.Can you authenticate the IP address of www.darpa.mil Explain why or why not.\\

  \textbf{We are given two records of RRSIG signed by `darpa.mil` so yes we can authenticate the IP address since we use the process described above and start by using the public key of `darpa.mil` to verify the signature of RRSIG and then follow the steps to authenticating the public key.} \\

  % QUESTION 5
  \item Given the unsigned zone file below, suppose the DNS administrator decides to deploy DNSSEC and sign the zone using  DNSSEC.  If a resolver queries the signed zone for the A record (IP address) of "server.example.com, what record would be sent to securely prove that there is no host called "server.example.com."? \\

  \textbf{The response should be the signed message "next name after ns.example.com is username.example.com" which proves there is not only a server.example.com but also that the next address after ns.example.com is username.example.com with nothing inbetween.} \\

\end{enumerate}
\end{document}
