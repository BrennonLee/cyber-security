\documentclass[12pt]{article}
\setlength{\oddsidemargin}{0in}
\setlength{\evensidemargin}{0in}
\setlength{\textwidth}{6.5in}
\setlength{\parindent}{0in}
\setlength{\parskip}{\baselineskip}
\usepackage{amsmath,amsfonts,amssymb, siunitx}
\usepackage{listings}
\usepackage{graphicx}
\usepackage{array}
\usepackage{fancyhdr}
\usepackage{listings}
\usepackage{flexisym}
\usepackage[table]{xcolor}
\usepackage[utf8]{inputenc}
\pagestyle{fancy}
%Code listing style named "mystyle"
\lstdefinestyle{mystyle}{
  basicstyle=\footnotesize,
  breakatwhitespace=false,
  breaklines=false,
  captionpos=b,
  keepspaces=false,
  numbers=left,
  numbersep=5pt,
  showspaces=false,
  showstringspaces=false,
  showtabs=false,
  tabsize=2
}
%"mystyle" code listing set
\lstset{style=mystyle}
\begin{document}
\lhead{{\bf CSCI 3403 \\ Homework 11} }
\rhead{{\bf Brennon Lee  \\ Fall 2018, CU-Boulder}}
\renewcommand{\headrulewidth}{0.4pt}
\vspace{-3mm}
\begin{enumerate}
  % QUESTION 1
  \item \textbf{Question Not in the Book:}Consider the SNORT rule:
  \vspace{-1em}

  {\color{blue}
  \begin{verbatim}
  alert tcp $HOME_NET any <> $EXTERNAL_NET 6666:7000
  (msg:"CHAT IRC message"; flow:established;
  content:"PRIVMSG "; nocase; classtype:policy-violation;
  sid:1463; rev:6;)
  \end{verbatim}
  }
   Explain what the  snort rule does by answering:
   \begin{enumerate}
     \item What type of connections would the rule apply to?
     \item What type of traffic is being monitored?
     \item Is there any additional requirement on the traffic?
   \end{enumerate} \\

  \textbf{Answer} \\

  % QUESTION 2
  \item \textbf{Review Question 8.4} Describe the three logical components of an IDS.\\

  \textbf{Answer} \\

  % QUESTION 3
  \item \textbf{Review Question 8.8} Explain the base-rate fallacy.\\

  \textbf{Answer} \\

  % QUESTION 4
  \item \textbf{Problem 8.2} In the context of an IDS, we define a false positive to be an alarm generated by an IDS in which the IDS alerts to a condition that is actually benign. A false negative occurs when an IDS fails to generate an alarm when an alert-worthy condition is in effect. Using the following diagram, depict two curves that roughly indicate false positives and false negatives, respectively:\\

  \textbf{Answer} \\

  % QUESTION 5
  \item \textbf{Problem 8.3} \\

  \textbf{Answer} \\

  % QUESTION 6
  \item \textbf{Problem 8.4} One of the non-payload options in Snort is flow. This option distinguishes between clients and servers. This option can be used to specify a match only for packets flowing in one direction (client to server or visa-versa) and can specify a match only on established TCP connections. Consider the following Snort rule: \\

  {\color{blue}
  \begin{verbatim}
  alert tcp $EXTERNAL_NET any -> $SQL_SERVERS $ORACLE_PORTS\
  (msg: ``ORACLE create database attempt:;\
  flow: to_server, established; content: ``create database”;
  nocase;\
  classtype: protocol-command-decode;)
  \end{verbatim}
  }

  \begin{enumerate}
  \item {What does this rule do?} \\

  \textbf{Answer} \\

  \item {Comment on the significance of this rule if the Snort devices is placed inside or outside of the external firewall.} \\

  \textbf{Answer} \\

  \end{enumerate}

  %Question 7
  \item \textbf{Problem 8.6}\\

  \textbf{Answer} \\


\end{enumerate}
\end{document}
