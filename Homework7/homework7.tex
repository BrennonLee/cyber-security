\documentclass[12pt]{article}
\setlength{\oddsidemargin}{0in}
\setlength{\evensidemargin}{0in}
\setlength{\textwidth}{6.5in}
\setlength{\parindent}{0in}
\setlength{\parskip}{\baselineskip}
\usepackage{amsmath,amsfonts,amssymb, siunitx}
\usepackage{graphicx}
\usepackage{array}
\usepackage{fancyhdr}
\usepackage{listings}
\usepackage{flexisym}
\usepackage[table]{xcolor}
\usepackage[utf8]{inputenc}
\pagestyle{fancy}
%Code listing style named "mystyle"
\lstdefinestyle{mystyle}{
  basicstyle=\footnotesize,
  breakatwhitespace=false,
  breaklines=false,
  captionpos=b,
  keepspaces=false,
  numbers=left,
  numbersep=5pt,
  showspaces=false,
  showstringspaces=false,
  showtabs=false,
  tabsize=2
}
%"mystyle" code listing set
\lstset{style=mystyle}
\begin{document}
\lhead{{\bf CSCI 3403 \\ Homework 7} }
\rhead{{\bf Brennon Lee  \\ Fall 2018, CU-Boulder}}
\renewcommand{\headrulewidth}{0.4pt}
\vspace{-3mm}
\begin{enumerate}
% QUESTION 1
\item \textbf{Review Question 10.2} List the three distinct types of locations in a process address space that buffer overflow attacks typically target. \\
% QUESTION 2
\item \textbf{Review Question 10.3} What are the possible consequences of a buffer overflow occuring? \\
% QUESTION 3
\item \textbf{Review Question 10.9} Describe what a NOP sled is and how it is used in a buffer overflow attack? \\
% QUESTION 4
\item \textbf{Problem 10.2} Rewrite the program shown in Figure 10.1a so it is no longer vulnerable to a stack buffer overflow. \\
% QUESTION 5
\item \textbf{Problem 10.3} Rewrite the program shown in Figure 10.5a so it is no longer vulnerable to a stack buffer overflow. \\
% QUESTION 6
\item \textbf{Problem 10.4}  Rewrite the program shown in Figure 10.7a so it is no longer vulnerable to a stack buffer overflow. \\
% QUESTION 7
\item \textbf{Review Question 11.3} List some possible sources of program input. \\
% QUESTION 8
\item \textbf{Review Question 11.6} Define a cross-site scripting attack. List an example of such an attack. \\
\end{enumerate}
\end{document}
